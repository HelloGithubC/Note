\documentclass[UTF8,12pt]{ctexart}
%\setmainfont{Noto Sans CJK SC}
\usepackage{xeCJK}
\xeCJKsetup{AutoFakeSlant=0.4}
\setCJKmainfont[BoldFont=.PingFang SC Semibold]{.PingFang SC Regular}
\usepackage{amsmath}
\usepackage{amssymb}
\usepackage{fourier}
\usepackage{bm}
\usepackage{geometry}
\geometry{a4paper, left=2cm, right=2cm,top=2cm,bottom=2cm}
\usepackage{pifont}
%\ding{172}=\textcircled{1}
\usepackage{booktabs}
\usepackage{hyperref}
\usepackage{mathrsfs}
\usepackage{hyperref}
\usepackage{graphicx}
\usepackage{wrapfig}
\usepackage{float}
\usepackage[perpage]{footmisc}
\newcommand*{\dif}{\mathop{}\!\mathrm{d}}
\renewcommand{\thefootnote}{\ding{\numexpr171+\value{footnote}}}

\title{Part1 Transform}
\author{Xiao Liang}
\date{\today}


\begin{document}
	\maketitle
	\numberwithin{equation}{section}
	%\CJKfontspec{Noto Sans Mono CJK SC}
	\CJKfontspec[BoldFont=.PingFang SC Semibold]{.PingFang SC Regular}

	\section{傅里叶级数}
	傅里叶级数本质上就是一种特殊形式的函数展开,类似泰勒定理,其基底函数取三角函数中的正弦和余弦函数。在傅里叶级数中,任意两个不同的基底函数在$ [0,2\pi] $的范围内是正交的。
	
	\subsection{周期函数的傅里叶级数}
	\subsubsection{一般情况}
	
	一个傅里叶级数在一般情况下表示为
	\begin{equation}
		f(x) = a_{0} + \sum_{n=1}^{\infty} (a_{n} \cos nx + b_{n} \sin nx)\label{equ_fourier}
	\end{equation}
	
\noindent 其中,$ a_{0},a_{n} \ and \ b_{n} $是展开系数。假定一个周期为$ 2 \pi $的函数$ f(x) $可以按照\autoref{equ_fourier}展开,则通过傅里叶级数的基底函数相互正交的性质以及在一个周期内$ \cos nx \ and \ \sin nx $积分为零的性质可以得到:
\begin{equation}
	\begin{aligned}
		a_{0} &= \frac{1}{2 \pi} \int_{0}^{2 \pi} f(x) \dif x\\
		a_{n} &= \frac{1}{\pi} \int_{0}^{2 \pi} f(x) \cos nx \dif x\\
		b_{n} &= \frac{1}{\pi} \int_{0}^{2 \pi} f(x) \sin nx \dif x
	\end{aligned}
\end{equation}

	\subsubsection{傅里叶级数的收敛性}
	\autoref{equ_fourier}的收敛性问题由狄利克雷(Dirichlet)定理描述,该定理的完整叙述是:
	
	假定
	
	(1) $ f(x) $在$ (-\pi,\pi) $内除了有限个点外有定义且单值;
	
	(2) $ f(x) $在$ (-\pi,\pi) $外是周期函数,周期为$ 2 \pi $;
	
	(3) $ f(x) $在$ (-\pi,\pi) $内分段连续,即$ f(x) $分段光滑,
	
	\noindent 则傅里叶级数收敛于
	\begin{equation}
	\begin{aligned}
	a_{0}+\sum_{n=1}^{\infty}\left(a_{n} \cos n x+b_{n} \sin n x\right)&=f(x)\quad (continuous\ point\ at\ x)\\
	a_{0}+\sum_{n=1}^{\infty}\left(a_{n} \cos n x+b_{n} \sin n x\right)&=\frac{f(x-0)+f(x+0)}{2}\quad (break\ point\ at\ x)
	\end{aligned}\label{equ_Dirichlet}
	\end{equation}
	
	狄利克雷条件是傅里叶级数的充分条件,但不是必要条件,在实际问题中这些条件通常是满足的。
	
	\subsubsection{傅里叶级数的推广}
	上述傅里叶级数可以推广到以$ 2L $为周期的函数,在这种情况下,\autoref{equ_fourier}变为
	\begin{equation}
	f(x)=a_{0}+\sum_{n=1}^{\infty}\left(a_{n} \cos \frac{n \pi}{L} x+b_{n} \sin \frac{n \pi}{L} x\right)
	\end{equation}
	
\noindent 相应的,展开系数为
\begin{equation}
\begin{aligned} a_{0} &=\frac{1}{2 L} \int_{-L}^{L} f(t) \mathrm{d} t \\ a_{n} &=\frac{1}{L} \int_{-L}^{L} f(t) \cos \frac{n \pi}{L} t \mathrm{d} t \\ b_{n} &=\frac{1}{L} \int_{-L}^{L} f(t) \sin \frac{n \pi}{L} t \mathrm{d} t \end{aligned}
\end{equation}

	\subsection{半幅傅里叶级数}
	在许多实际问题中,函数$ \phi (x) $是一个定义在有限区间$ 0<x<L $上的任意函数,因为它不具有周期性,上一节的结果是不适用的。这样的函数能够展开为半幅傅里叶级数(half-range Fourier series)。
	
	如果$ \phi (x) $在$ 0<x<L $内是分段光滑的,则$ \phi (x) $有正弦函数展开式
	\begin{equation}
	\phi(x)=\sum_{n=1}^{\infty} C_{n} \sin \frac{n \pi x}{L}
	\end{equation}
	
\noindent 和余弦函数展开式
\begin{equation}
\phi(x)=D_{0}+\sum_{n=1}^{\infty} D_{n} \cos \frac{n \pi x}{L}
\end{equation}

\noindent 其中,展开系数为
\begin{equation}
	\begin{aligned}
	C_{n}&=\frac{2}{L} \int_{0}^{L} \phi(x) \sin \frac{n \pi x}{L} \mathrm{d} x \quad(n=1,2,3, \cdots)\\
	D_{0}&=\frac{1}{L} \int_{0}^{L} \phi(x) \mathrm{d} x\\
	D_{n}&=\frac{2}{L} \int_{0}^{L} \phi(x) \cos \frac{n \pi x}{L} \mathrm{d} x \quad(n=1,2,3, \cdots)
	\end{aligned}
\end{equation}

	展开系数的推导方式同上一节的展开系数的推导,首先得到
	\begin{equation}
	\int_{0}^{L} \sin \frac{m \pi x}{L} \sin \frac{n \pi x}{L} \mathrm{d} x=\frac{L}{2} \delta_{m n}
	\end{equation}
	
\noindent 然后可以证明
\begin{equation}
\begin{aligned}
\int_{0}^{L} \phi(x) \sin \frac{m \pi x}{L} \mathrm{d} x&=\int_{0}^{L} \sin \frac{m \pi x}{L}\left(\sum_{n=1}^{\infty} C_{n} \sin \frac{n \pi x}{L}\right) \mathrm{d} x\\
&=\frac{L}{2} \sum_{n=1}^{\infty} C_{n} \delta_{m n}=\frac{L}{2} C_{m}
\end{aligned}
\end{equation}

\noindent 以及
\begin{equation}
\int_{0}^{L} \phi(x) \mathrm{d} x=D_{0} \int_{0}^{L} \mathrm{d} x+\sum_{n=1}^{\infty} D_{n} \underbrace{\int_{0}^{L} \cos \frac{n \pi x}{L} \mathrm{d} x}_{=0}
\end{equation}

\noindent 剩下那个完全是照葫芦画瓢,这里不赘述。

	半幅傅里叶级数的收敛性也服从狄利克雷条件\autoref{equ_Dirichlet}.
	
	对于具体问题,特别是数学物理方法的不同边值问题,半幅傅里叶级数呈现不同的形式,除了标准形式外,还可以取
	\begin{equation}
	\begin{aligned}
	\phi(x)&=\sum_{n=0}^{\infty} C_{n} \sin \frac{(2 n+1) \pi x}{2 L}\\
	\phi(x)&=\sum_{n=0}^{\infty} D_{n} \cos \frac{(2 n+1) \pi x}{2 L}
	\end{aligned}
	\end{equation}
	
	\noindent 的形式,其中的展开系数为
	\begin{equation}
	\begin{aligned} C_{n} &=\frac{2}{L} \int_{0}^{L} \phi(x) \sin \frac{(2 n+1) \pi x}{2 L} \mathrm{d} x \\ D_{n} &=\frac{2}{L} \int_{0}^{L} \phi(x) \cos \frac{(2 n+1) \pi x}{2 L} \mathrm{d} x \end{aligned}
	\end{equation}
	
	\subsection{傅里叶积分}
	本节讨论一个定义在$ (-\infty,\infty) $区间的非周期函数的傅里叶展开问题。傅里叶级数扩展到连续变化的情况即是傅里叶积分。
	
	考虑一个满足绝对可积条件的周期函数,周期为$ 2L $,则它的傅里叶级数为
	\begin{equation}
	f(x)=a_{0}+\sum_{n=1}^{\infty}\left(a_{n} \cos \frac{n \pi}{L} x+b_{n} \sin \frac{n \pi}{L} x\right)\label{equ_sum}
	\end{equation}
	
	现在来对该级数进行操作,使其成为定义在$ (-\infty, \infty) $的非周期函数,最简单的途径是令$ L \rightarrow \infty $,则
	\begin{equation}
	a_{0}=\frac{1}{2 L} \int_{-L}^{L} f(t) \mathrm{d} t \stackrel{L \rightarrow \infty}{\longrightarrow} 0
	\end{equation}
	
\noindent 其中用到了绝对可积函数的性质($ f(x) \rightarrow 0 $,当$ x \rightarrow \pm \infty $)。进一步,令$ \omega_{n} = \frac{n \pi}{L} $,则
\begin{equation}
\Delta \omega=\omega_{n}-\omega_{n-1}=\frac{\pi}{L}
\end{equation}

\noindent 求和间隔$ \Delta \omega $在$ L \rightarrow \infty $时将变成微元$ \dif \omega $,而变量$ \omega_{n} $趋于连续变化,\textbf{傅里叶级数则转变成积分的形式},即
\begin{equation}
\sum_{n=1}^{\infty} \cdots \Delta \omega \stackrel{L \rightarrow \infty}{\longrightarrow} \int_{0}^{\infty} \cdots \mathrm{d} \omega
\end{equation}

	在这样的基础上,可以得到
	\begin{equation}
	\begin{aligned}
	\sum_{n=1}^{\infty} a_{n} \cos \frac{n \pi}{L} x&=\sum_{n=1}^{\infty} \frac{1}{L}\left[\int_{-L}^{L} f(t) \cos \frac{n \pi}{L} t \mathrm{d} t\right] \cos \frac{n \pi}{L} x\\
	&\stackrel{L \rightarrow \infty}{\longrightarrow} \int_{0}^{\infty} \mathrm{d} \omega\left[\frac{1}{\pi} \int_{-\infty}^{\infty} f(t) \cos \omega t \mathrm{d} t\right] \cos \omega x
	\end{aligned}
	\end{equation}
	
\noindent 同理
\begin{equation}
\sum_{n=1}^{\infty} b_{n} \sin \frac{n \pi}{L} x \stackrel{L \rightarrow \infty}{\longrightarrow} \int_{0}^{\infty} \mathrm{d} \omega\left[\frac{1}{\pi} \int_{-\infty}^{\infty} f(t) \sin \omega t \mathrm{d} t\right] \sin \omega x
\end{equation}

\noindent 由此可以将\autoref{equ_sum}写成
\begin{equation}
f(x)=\int_{0}^{\infty}[A(\omega) \cos \omega x+B(\omega) \sin \omega x] \mathrm{d} \omega\label{equ_int}
\end{equation}

\noindent 其中
\begin{equation}
\begin{aligned}
A(\omega)&=\frac{1}{\pi} \int_{-\infty}^{\infty} f(t) \cos \omega t \mathrm{d} t \\
 B(\omega)&=\frac{1}{\pi} \int_{-\infty}^{\infty} f(t) \sin \omega t \mathrm{d} t
 \end{aligned}\label{equ_int_C}
\end{equation}
\noindent \autoref{equ_int}就是函数$ f(x) $的傅里叶积分表示,其中的展开系数由\autoref{equ_int_C}确定。

	在上述由级数到积分的转变中,具体的变化是:\ding{172} $L \rightarrow \infty$ 和$ f(x) $ 的绝对可积性质导致系数$ a_{0} = 0 $;\ding{173} 计算系数的积分限发生了变化,变为全实数范围进行积分;\ding{174} 分立变化的求和变量变成了连续变化的积分变量,而确保级数可以转变为积分的条件而是绝对可积条件,这个条件进一步确保了系数的存在性。

	为了使傅里叶积分式\autoref{equ_int}具有\autoref{equ_Dirichlet}所确定的收敛行为,则需要满足狄利克雷定理成立的条件。

	物理上通常认为,$ f(x) $代表一个“信号”,而系数$ A (\omega) $和$ B(\omega) $则是信号$ f(x) $的频谱分布函数,它们分别相应于正交分量$ \cos \omega t $和$ \sin \omega t $.由信号到频谱的过程,通常称为\textbf{傅里叶分析}。

	绝对可积条件是傅里叶积分存在的充分条件,但不是必要条件。

	\subsection{例题}
	1、求下面的函数的傅里叶积分表示式
	\begin{equation}
	f(x)=\left\{\begin{array}{ll}{1} & {(|x| \leqslant 1)} \\ {0} & {(|x|>1)}\end{array}\right.\label{equ_example_3}
	\end{equation}

	首先计算频谱分布函数,考虑到\autoref{equ_example_3}表示一个偶函数,故$ B(\omega) =0 $,而
	\begin{equation}
		A(\omega)=\frac{1}{\pi} \int_{-1}^{1} \cos \omega t \mathrm{d} t=\frac{2 \sin \omega}{\pi \omega}
		\end{equation}
\noindent 于是傅里叶积分表示式为
\begin{equation}
	f(x)=\frac{2}{\pi} \int_{0}^{\infty} \frac{\sin \omega \cos \omega x}{\omega} \mathrm{d} \omega
	\end{equation}

\noindent 然后根据狄利克雷定理可以得到
\begin{equation}
\frac{2}{\pi} \int_{0}^{\infty} \frac{\sin \omega \cos \omega x}{\omega} \mathrm{d} \omega=\left\{\begin{array}{ll}{1} & {(|x|<1)} \\ {\frac{1}{2}} & {(|x|=1)} \\ {0} & {(|x|>1)}\end{array}\right.
\end{equation}

\noindent 特别地,在$ x=0 $,有
\begin{equation}
	\int_{0}^{\infty} \frac{\sin \omega}{\omega} \mathrm{d} \omega=\frac{\pi}{2}
	\end{equation}

	2、求下列函数的傅里叶积分表达式
	\begin{equation}
	f(x)=\left\{\begin{array}{ll}{\cos x} & {(|x|<\pi / 2)} \\ {0} & {(|x|>\pi / 2)}\end{array}\right.
	\end{equation}

	首先按照套路计算频谱分布函数,由于是偶函数,$ B (\omega) = 0 $,而
	\begin{equation}
		\begin{aligned}
			A(\omega)&=\frac{1}{\pi} \int_{-\infty}^{\infty} f(x) \cos \omega x \mathrm{d} x=\frac{2}{\pi} \int_{0}^{\pi / 2} \cos x \cos \omega x \mathrm{d} x \\
			&=\frac{2 \cos (\omega \pi / 2)}{\pi\left(1-\omega^{2}\right)}
		\end{aligned}
		\end{equation}
	
\noindent 这个结果允许$ \omega = 1 $,这时可以使用洛必达法则,得到
\begin{equation}
	\lim _{\omega \rightarrow 1} \frac{2 \cos (\omega \pi / 2)}{\pi\left(1-\omega^{2}\right)}=\frac{1}{2} \lim _{\omega \rightarrow 1} \frac{\sin (\omega \pi / 2)}{\omega}=\frac{1}{2}
	\end{equation}

\noindent 由此得到该函数的傅里叶积分表达式为
\begin{equation}
	f(x)=\frac{2}{\pi} \int_{0}^{\infty} \frac{\cos (\omega \pi / 2)}{1-\omega^{2}} \cos \omega x d \omega
	\end{equation}

\noindent 在实数范围内没有间断点,所以可以得到
\begin{equation}
\frac{2}{\pi} \int_{0}^{\infty} \frac{\cos (\omega \pi / 2)}{1-\omega^{2}} \cos \omega x \mathrm{d} \omega=\left\{\begin{array}{ll}{\cos x} & {(|x|<\pi / 2)} \\ {0} & {(|x|>\pi / 2)}\end{array}\right.
\end{equation}

noindent 特别地,在$ x = 0 $处,有
\begin{equation}
	\int_{0}^{\infty} \frac{\cos (\omega \pi / 2)}{1-\omega^{2}} \mathrm{d} \omega=\frac{\pi}{2}
	\end{equation}


	\section{傅里叶变换}
	傅里叶变换是基于傅里叶积分的一类变换,本质上是把一个函数经过积分运算变为另一个函数,两个函数具有一一对应性,由此可以简化运算和得到另一个方面的对应的信息。
	
	\subsection{傅里叶变换}
\end{document}
